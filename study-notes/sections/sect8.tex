\section{Risk Measures}
\label{sect:risk-measures}
\subsection{Introduction}
\begin{enumerate}
\item For a loss \faIcon{fire-alt}, we use a nonnegative loss random variable
\(X\) to measure it. Now, consider a \emph{risk}, which is an exposure to
possibility of loss \faIcon{fire-alt}. The main goal in this section is to
study how to \emph{measure} a \emph{risk} (quantifying exposure/risk
``level'').  It turns out that, unlike a loss, there are multiple ways to
measure a risk.

\item A \defn{risk measure} is a function \(\rho:\mathcal{X}\to\R\) where
\(\mathcal{X}\) is the set of all possible nonnegative loss random variables.
For any loss random variable \(X\), we assign a number \(\rho(X)\) to it, which
serves as a measure for the risk corresponding to that loss.

The value \(\rho(X)\) is to be interpreted as the amount of \emph{capital}
needed for the insurer \faIcon{building} to \emph{protect} \faIcon{shield-alt}
against the loss \(X\). Then, larger value of \(\rho(X)\) indicates higher
risk.

\begin{note}
This interpretation is more natural if the function \(\rho\) satisfies certain
properties, which will be discussed in \cref{subsect:coherent-risk-measures}.
However, these properties are not \emph{required} in the definition.
\end{note}
\end{enumerate}
\subsection{Premium Principles}
\begin{enumerate}
\item The first use of risk measures in actuarial science was the development
of \emph{premium principles} (e.g., equivalence principle in STAT3901). The
premium principles are applied to a loss distribution (distribution of loss
\(X\)) for determining a suitable premium to charge for insuring the loss
\(X\).

\item Some examples of premium principles expressed in the language of risk
measures are:
\begin{itemize}
\item \defn{expected value premium principle}: \(\rho(X)=(1+k)\expv{X}\)
for some \(k\ge 0\)
\item \defn{standard deviation premium principle}: \(\rho(X)=\expv{X}+k\sqrt{\vari{X}}\)
for some \(k\ge 0\)
\item \defn{variance premium principle}: \(\rho(X)=\expv{X}+k\vari{X}\)
for some \(k\ge 0\)
\end{itemize}
\begin{note}
To follow one of these premium principle, the premium to be charged for
insuring the loss \(X\) is \(\rho(X)\).
\end{note}

\item Each of these premium principles gives a premium that is \emph{at least}
the expected loss \(\expv{X}\). The excess amount serves as a ``cushion''
against adverse experience. Such excess amount is known as \defn{premium
loading}:
\[
\text{premium loading}=\text{premium}-\expv{X}.
\]
In the standard deviation and variance principles, the premium loading is
related to the variance of the loss \(X\):
\begin{itemize}
\item standard deviation premium principle: premium loading = \(\alpha\sqrt{\vari{X}}\)
\item variance premium principle: premium loading = \(\alpha\vari{X}\)
\end{itemize}
\end{enumerate}
\subsection{Value-at-Risk}
\begin{enumerate}
\item A popular risk measure is the \emph{value-at-risk} (VaR). Let
\(\alpha\in(0,1)\). \begin{warning}
Here 0 and 1 are \emph{excluded}!
\end{warning}
The \defn{value-at-risk at confidence level \(\alpha\)} (or
\defn{\(\alpha\)-VaR}) of a loss \(X\) is the \(\alpha\)th quantile (or
\(100\alpha\)\% percentile) of \(X\):
\[
\vatr{\alpha}{X}=\min\{x\ge 0:F_X(x)\ge \alpha\}
\]
where \(F_X\) is the cdf of \(X\).
\begin{center}
\begin{tikzpicture}
\begin{axis}[domain=0:5, axis lines=middle, samples=100,
ymax=3.5,
ytick={\fpeval{3^(1/2)},\fpeval{5-(4-3^(1/2))}}, yticklabels={\(\alpha\),1},
xtick={3}, xticklabels={\(\vatr{\alpha}{X}\)},
ylabel={\(F_X(x)\)},xlabel={\(x\)},
xlabel style={anchor=west},
ylabel style={anchor=south},
]
\addplot[blue, domain=0:3]{x^(1/2)};
\addplot[blue, domain=3:4]{3^(1/2)};
\addplot[blue, domain=4:5]{x-(4-3^(1/2))};
\draw[yellow, opacity=0.5, line width=0.2cm] (3,\fpeval{3^(1/2)}) -- (4,\fpeval{3^(1/2)});
\draw[yellow, opacity=0.5, line width=0.2cm] (4,\fpeval{3^(1/2)}) -- (5,\fpeval{5-(4-3^(1/2))});
\draw[yellow, opacity=0.5, line width=0.2cm] (3,0) -- (5,0);
\draw[-Latex] (3.5,0.5) -- (3,0.1)
node[pos=-0.3]{minimum};
\end{axis}
\end{tikzpicture}
\end{center}
\begin{remark}
\item In most cases, \(\alpha\)-VaR is the value that is not exceeded by the
loss \(X\) with probability \(\alpha\), i.e.,
\(\prob{X\le\vatr{\alpha}{X}}=\alpha\). (See
\labelcref{it:alpha-cdf-no-intersect} for an example where this is not the
case.)
\item It can be interpreted as the amount of capital required to ensure that
the loss can be ``absorbed'' by the insurer \faIcon{building} (so that
\faIcon{building} would not bankrupt) with a high degree of certainty (when
\(\alpha\) is large).
\item If \(F_X\) is continuous and strictly increasing, we have
\(\vatr{\alpha}{X}=F_X^{-1}(\alpha)\).
\end{remark}
\item \label{it:alpha-cdf-no-intersect}
If \(F_X\) is continuous, then we always have
\(F_X(\vatr{\alpha}{X})=\alpha\). Thus, \(\alpha\)-VaR is the value that is not
exceeded by the loss \(X\) with probability \(\alpha\). However, this is not
necessarily the case if \(F_X\) is not continuous:

\begin{center}
\begin{tikzpicture}
\begin{axis}[domain=0:5, axis lines=middle, samples=100,
ymax=5.5,
ytick={2.2,5,3}, yticklabels={\(\alpha\),1,\(F_X(\vatr{\alpha}{X})\)},
xtick={3}, xticklabels={\(\vatr{\alpha}{X}\)},
ylabel={\(F_X(x)\)},xlabel={\(x\)},
xlabel style={anchor=west},
ylabel style={anchor=south},
]
\addplot[blue, domain=0:3]{x^(1/2)};
\addplot[blue, domain=3:5]{x};
\draw[blue] (3,\fpeval{3^(1/2)}) circle [radius=0.8mm];
\draw[blue, fill] (3,3) circle [radius=0.8mm];
\addplot[violet, dashed, domain=0:5]{2.2};
\end{axis}
\end{tikzpicture}
\end{center}
In this case, we have \(F_X(\vatr{\alpha}{X})>\alpha\).
\end{enumerate}
\subsection{Conditional Tail Expectation}
\begin{enumerate}
\item Let \(\alpha\in(0,1)\). The \defn{conditional tail expectation at
confidence level \(\alpha\)} (or \defn{\(\alpha\)-CTE}) is the expected loss given
that the loss exceeds its \(\alpha\)-VaR:
\[
\cte{\alpha}{X}=\expv{X|X>\vatr{\alpha}{X}}.
\]
\begin{center}
\begin{tikzpicture}
\begin{axis}[domain=0:10, samples=100, axis lines=middle,
xtick={8}, xticklabels={\(\vatr{\alpha}{X}\)},
ytick=\empty, xmin=-0.1]
\addplot[blue, name path=A]{x^2*e^(-x)};
\addplot[draw=none, name path=B]{0};
\addplot[red, opacity=0.3] fill between[of=A and B, soft clip={domain=8:10}];
\draw[-Latex] (7,0.25) -- (9,0.01)
node[pos=-0.1]{``average loss at tail''};
\end{axis}
\end{tikzpicture}
\end{center}
\begin{note}
From \labelcref{it:alpha-cdf-no-intersect}, the probability
\(\prob{X>\vatr{\alpha}{X}}\) is \(1-\alpha\) if \(F_X\) is continuous, and it
is not necessarily \(1-\alpha\).
\end{note}
\end{enumerate}
\subsection{Tail Value-at-Risk}
\begin{enumerate}
\item Let \(\alpha\in(0,1)\). The \defn{tail value-at-risk at confidence level
\(\alpha\)} (or \defn{\(\alpha\)-TVaR}) of the loss \(X\) is the ``average'' of
the VaR at confidence level not less than \(\alpha\):
\[
\tvar{\alpha}{X}=\frac{1}{1-\alpha}\int_{\alpha}^{1}\vatr{p}{X}\dd{p}.
\]
\begin{center}
\begin{tikzpicture}[
declare function={
f(\x)=(\x<1.204)*0.7 + (\x>=1.204)*(1-e^(-x));
g(\x)=(\x<1.609)*0.8 + (\x>=1.609)*(1-e^(-x));
}
]
\begin{axis}[domain=0:3, axis lines=middle,
ymax=1.1,
ytick={1,0.7, 0.8, 0.85}, yticklabels={1, \(\alpha\), \(p\), \(p+\dd{p}\)},
xtick={1.609}, xticklabels={\(\vatr{p}{X}\)},
xlabel=\(t\), ylabel=\(F_X(t)\),
xlabel style={anchor=west},
ylabel style={anchor=south},
samples=100]
\addplot[blue]{1-e^(-x)};
\addplot[draw=none, name path=A]{f(x)};
\addplot[draw=none, name path=B] {1};
\addplot[draw=none, name path=C]{g(x)};
\addplot[draw=none, name path=D] {0.85};
\addplot[brown, opacity=0.3] fill between[of=A and B];
\addplot[magenta, opacity=0.3] fill between[of=C and D, soft clip={domain=0:1.897}];
\draw[-Latex] (1.2,0.5) -- (0.5,0.75)
node[pos=-0.4, brown, font=\small]{area = \(\displaystyle \int_{\alpha}^{1}\vatr{p}{X}\dd{p}=(1-\alpha)\tvar{\alpha}{X}\)};
\draw[-Latex] (1.7,0.6) -- (0.8,0.82)
node[pos=-0.4, magenta]{area \(\approx \vatr{p}{X}\dd{p}\)};
\draw[yellow, opacity=0.5, line width=0.2cm] (1.204,0) -- (3,0);
\draw[yellow, opacity=0.5, line width=0.2cm] (0,0.7) -- (0,1);
\addplot[yellow, opacity=0.5, domain=1.204:3, line width=0.2cm, line join=round]{1-e^(-x)};
\end{axis}
\end{tikzpicture}
\end{center}
\begin{note}
We consider VaR at confidence level in the ``right tail''.
\end{note}

\item The following are several more cases represented geometrically:
\begin{center}
\begin{tikzpicture}[
declare function={
f(\x)=(\x<3)*2.2 + (\x>=3)*(\x);
}
]
\begin{axis}[domain=0:5, axis lines=middle, samples=500,
ymax=5.5,
ytick={2.2,5}, yticklabels={\(\alpha\),1},
xtick={3}, xticklabels={\(\vatr{\alpha}{X}\)},
ylabel={\(F_X(x)\)},xlabel={\(x\)},
xlabel style={anchor=west},
ylabel style={anchor=south},
]
\addplot[blue, domain=0:3]{x^(1/2)};
\addplot[blue, domain=3:5]{x};
\addplot[draw=none, name path=A]{f(x)};
\addplot[draw=none, name path=B] {5};
\addplot[brown, opacity=0.3] fill between[of=A and B];
\draw[blue] (3,\fpeval{3^(1/2)}) circle [radius=0.8mm];
\draw[blue, fill] (3,3) circle [radius=0.8mm];
\addplot[violet, dashed, domain=0:5]{2.2};
\draw[-Latex] (3,1.5) -- (1.6,2.75)
node[pos=-0.4, brown]{area = \(\displaystyle \int_{\alpha}^{1}\vatr{p}{X}\dd{p}\)};
\end{axis}
\end{tikzpicture}

\begin{tikzpicture}[
declare function={
f(\x)=(\x<4)*1.732 + (\x>=4)*(\x-2.268);
}
]
\begin{axis}[domain=0:5, axis lines=middle, samples=100,
ymax=3.5,
ytick={\fpeval{3^(1/2)},\fpeval{5-(4-3^(1/2))}}, yticklabels={\(\alpha\),1},
xtick={4}, xticklabels={\(\vatr{\alpha^+}{X}\)},
ylabel={\(F_X(x)\)},xlabel={\(x\)},
xlabel style={anchor=west},
ylabel style={anchor=south},
]
\addplot[blue, domain=0:3]{x^(1/2)};
\addplot[blue, domain=3:4]{3^(1/2)};
\addplot[blue, domain=4:5]{x-(4-3^(1/2))};
\addplot[draw=none, name path=A]{f(x)};
\addplot[draw=none, name path=B] {5-2.268};
\addplot[brown, opacity=0.3] fill between[of=A and B];
\draw[-Latex] (3,1.5) -- (1.6,2.25)
node[pos=-0.4, brown]{area = \(\displaystyle \int_{\alpha}^{1}\vatr{p}{X}\dd{p}\)};
\end{axis}
\end{tikzpicture}
\end{center}
\begin{note}
\(\alpha^+\) represents a value that is ``just'' larger than \(\alpha\).
\end{note}
\end{enumerate}
\subsection{Expected Shortfall}
\begin{enumerate}
\item Let \(\alpha\in(0,1)\). The \defn{expected shortfall at confidence level
\(\alpha\)} (or \defn{\(\alpha\)-ESF}) of the loss \(X\) is the expected value
of shortfall (with respect to \(\alpha\)-VaR):
\[
\esf{\alpha}{X}=\expv{(X-\vatr{\alpha}{X})_{+}}.
\]
\begin{note}
The shortfall with respect to \(\alpha\)-VaR is the excess amount of loss over
the \(\alpha\)-VaR (capital reserved based on VaR) when the loss exceeds the
\(\alpha\)-VaR, and is zero otherwise.
\end{note}

\item By \cref{prp:mean-sl-surv}, we can express \(\esf{\alpha}{X}\) as:
\[
\esf{\alpha}{X}=\int_{\vatr{\alpha}{X}}^{\infty}[1-F_X(x)]\dd{x}.
\]
We can represent the expected shortfall \(\esf{\alpha}{X}\) geometrically as
follows:
\begin{center}
\begin{tikzpicture}
\begin{axis}[domain=0:3, axis lines=middle,
ymax=1.1,
ytick={1,0.7}, yticklabels={1,\(\alpha\)},
xtick={1.204}, xticklabels={\(\vatr{\alpha}{X};\)},
xlabel={\(t\)}, ylabel={\(F_X(t)\)},
xlabel style={anchor=west},
ylabel style={anchor=south}]
\addplot[blue, name path=A]{1-e^(-x)};
\addplot[draw=none, name path=B]{1};
\addplot[brown, opacity=0.3] fill between[of=A and B, soft clip={domain=1.204:3}];
\draw[-Latex] (2,0.7) -- (1.5,0.9)
node[pos=-0.2, brown]{area = \(\esf{\alpha}{X}\)};
\end{axis}
\end{tikzpicture}
\end{center}
\end{enumerate}
\subsection{Relationships Between VaR, CTE, TVaR and ESF}
\begin{enumerate}
\item For any \(\alpha\in(0,1)\), we have:
\begin{enumerate}
\item \label{it:tvar-var-esf}
\(\displaystyle \tvar{\alpha}{X}=\boxed{\vatr{\alpha}{X}+\frac{1}{1-\alpha}\cdot \esf{\alpha}{X}}\)

\begin{pf}
\begin{center}
\begin{tikzpicture}
\begin{axis}[domain=0:3, axis lines=middle,
ymax=1.5,
ytick={1,0.7}, yticklabels={1,\(\alpha\)},
xtick={1.204}, xticklabels={\(\vatr{\alpha}{X}\)},
xlabel={\(t\)}, ylabel={\(F_X(t)\)},
xlabel style={anchor=west},
ylabel style={anchor=south}]
\addplot[blue, name path=A]{1-e^(-x)};
\addplot[draw=none, name path=B]{1};
\addplot[draw=none, name path=C]{0.7};
\addplot[brown, opacity=0.3] fill between[of=A and B, soft clip={domain=1.204:3}];
\addplot[violet, opacity=0.3] fill between[of=C and B, soft clip={domain=0:1.204}];
\draw[-Latex] (2,0.7) -- (1.5,0.9)
node[pos=-0.2, brown]{area = \(\esf{\alpha}{X}\)};
\draw[-Latex] (1,0.5) -- (0.6,0.75)
node[pos=-0.4, violet]{area = \((1-\alpha)\vatr{\alpha}{X}\)};
\draw[very thick, decorate,decoration={calligraphic brace, amplitude=5pt, raise=5pt}] (0,1) -- (3,1)
node[midway, above=0.5cm]{area = \((1-\alpha)\tvar{\alpha}{X}\)};
\end{axis}
\end{tikzpicture}
\end{center}
From this, we see that
\[
(1-\alpha)\tvar{\alpha}{X}
={\color{violet}(1-\alpha)\vatr{\alpha}{X}}+{\color{brown}\esf{\alpha}{X}},
\]
which implies that
\[
\tvar{\alpha}{X}=\vatr{\alpha}{X}+\frac{1}{1-\alpha}\cdot \esf{\alpha}{X}.
\]
(The argument is similar when the cdf has other shape.)
\end{pf}

\item \label{it:cte-tvar}
\(\displaystyle \cte{\alpha}{X}=\boxed{\tvar{F_X(\vatr{\alpha}{X})}{X}}\) (when \(0<F_X(\vatr{\alpha}{X})<1\))

\begin{pf}
We have
\begin{align*}
\cte{\alpha}{X}=\expv{X|X>\vatr{\alpha}{X}}
&=\vatr{\alpha}{X}+\underbrace{\expv{X-\vatr{\alpha}{X}|X>\vatr{\alpha}{X}}}_{e_X(\vatr{\alpha}{X})} \\
&\overset{\labelcref{it:mrl-exp-sl-relation}}{=}\vatr{\alpha}{X}+\frac{\esf{\alpha}{X}}{1-F_X(\vatr{\alpha}{X})}\\
&=\vatr{\color{violet}F_X(\vatr{\alpha}{X})}{X}+\frac{1}{1-{\color{violet}F_X(\vatr{\alpha}{X})}}
\cdot \esf{\color{violet}F_X(\vatr{\alpha}{X})}{X}\\
&\overset{\labelcref{it:tvar-var-esf}}{=}\tvar{\color{violet}F_X(\vatr{\alpha}{X})}{X}.
\end{align*}
For the second-to-last equality, see the proof of \labelcref{it:cte-var-esf}.
\end{pf}

\item \label{it:cte-var-esf}
\(\displaystyle \cte{\alpha}{X}=\boxed{\vatr{\alpha}{X}+\frac{1}{1-F_X(\vatr{\alpha}{X})}\cdot \esf{\alpha}{X}}\) 
(when \(F_X(\vatr{\alpha}{X})<1\))

\begin{pf}
Apply \labelcref{it:cte-tvar} followed by \labelcref{it:tvar-var-esf}, and note that
\[
\vatr{F_X(\vatr{\alpha}{X})}{X}=\min\{{\color{violet}x}\ge 0:F_X({\color{violet}x})\ge F_X({\color{violet}\vatr{\alpha}{X}})\}
={\color{violet}\vatr{\alpha}{X}},
\]
and
\[
\esf{F_X(\vatr{\alpha}{X})}{X}
=\expv{\qty[X-{\color{brown}\vatr{F_X(\vatr{\alpha}{X})}{X}}]_{+}}
=\expv{\qty[X-{\color{brown}\vatr{\alpha}{X}}]_{+}}
=\esf{\alpha}{X}.
\]
\end{pf}
\end{enumerate}
\begin{note}
If \(F_X(\vatr{\alpha}{X})=\alpha\) (which is the case when \(F_X\) is
continuous), then we can simplify the formula in \labelcref{it:cte-tvar} to:
\[\cte{\alpha}{X}=\boxed{\tvar{\alpha}{X}}.\]
This shows CTE and TVaR are equivalent in this case!
\end{note}
\end{enumerate}
\subsection{Coherent Risk Measures}
\label{subsect:coherent-risk-measures}
\begin{enumerate}
\item When the risk measure \(\rho\) satisfies certain properties, the
interpretation of \(\rho(X)\) as amount of capital needed for the insurer to
protect against the loss \(X\) is more natural.
\item The properties are as follows. \begin{note}
The letters \(X\) and \(Y\) denote arbitrary nonnegative loss random variables
(in the set \(\mathcal{X}\)) in the following.
\end{note}
\begin{itemize}
\item \defn{translation invariance} (TI):
\[
\rho(X+c)=\rho(X)+c\quad\text{for any constant \(c\)}.
\]
\begin{note}
This means adding a constant amount of loss (positive or negative) implies
addition of the same amount to the required capital for protecting that loss.
\end{note}
\item \defn{positive homogeneity} (PH):
\[
\rho(\lambda X)=\lambda\rho(X)\quad\text{for any constant \(\lambda>0\)}.
\]
\begin{note}
This means changing the unit of loss only leads to the corresponding unit
change for the amount of capital required (but not change in the ``actual''
amount).\footnote{For example, suppose that the exchange rate is 1 USD = 160
JPY. Let \(X\) be the loss in USD. Suppose the amount of capital required in
USD is \(\rho(X)=100\text{ (USD)}\).  Changing the currency unit to JPY, the
loss becomes \(160X\) (in JPY), and the capital required in JPY should be
\[
160\rho(X)=16000\text{ (JPY)}
\]
in order to have no change in the ``actual'' amount. Hence, we should have
\[
\rho(160X)=160\rho(X).
\]}
\end{note}
\item \defn{monotonicity} (M):
\[
\prob{X\le Y}=1\implies \rho(X)\le\rho(Y).
\]
\begin{note}
This means when a loss \(Y\) is not less than another loss \(X\) (with
probability 1), the required capital for protecting the loss \(Y\) should also
be not less than that for the loss \(X\).
\end{note}
\item \defn{subadditivity} (S):
\[
\rho(X+Y)\le\rho(X)+\rho(Y).
\]
\begin{note}
This means diversification (adding/combining different losses together) cannot
possibly make the resulting \emph{total} amount of capital required greater
than what we would have if there were no diversification.
\end{note}
\end{itemize}
A risk measure satisfying TI, PH, M, and S is called a \defn{coherent risk
measure}.

\item \label{it:coherence-summary}
The following summarizes the coherence of some common risk measures.
\begin{center}
\begin{tabular}{cc}
\toprule
Risk measure&Coherent?\\
\midrule
\(\expv{\cdot}\)&yes\footnote{TI, PH, and S follow from linearity
of expectation. M follows from monotonicity of expectation.}\\
expected value premium principle&no (fails TI) (when \(k>0\))\\
standard deviation premium principle&no (fails M) (when \(k>0\))\\
variance premium principle&no (fails M) (when \(k>0\))\\
VaR&no (fails S)\\
CTE&no (fails S)\\
ESF&no (fails S)\\
TVaR&yes\footnote{It turns out to be highly non-trivial to prove this.}\\
\bottomrule
\end{tabular}
\end{center}
\end{enumerate}
