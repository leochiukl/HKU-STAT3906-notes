\section{Aggregate Loss Model}
\label{sect:agg-loss-model}
\begin{enumerate}
\item Previously, when multiple losses are involved, we consider them in
isolation. Here, we study the behaviour of their \emph{sum} (called
\defn{aggregate loss}).
\item A practical setting where such sum arises is when an insurer sells many
insurance policies \faIcon{arrow-right} pooling many risks and having
``diversification''.
\end{enumerate}
\subsection{Collective Risk Model}
\begin{enumerate}
\item In the \defn{collective risk model}, the aggregate loss is a sum
\(S\) of \(N\) (random variable) individual losses with amounts
\(X_1,\dotsc,X_N\):
\[
S=X_1+\dotsb+X_N.
\]
Furthermore we impose the following:
\begin{itemize}
\item When \(N=0\), we have \(S=0\).
\item \(X_1,X_2,\dotsc\) are i.i.d.\ random variables (with the same
distribution as a random variable \(X\)) and are independent of \(N\).
\end{itemize}
\begin{remark}
\item This setting is similar to \labelcref{it:cpd-dist-as-total-claims}.
\item The random variable \(N\) is called \defn{claim count} (or \defn{claim
frequency}).
\item Each random variable \(X_j\) is called \defn{severity}/\defn{single
loss}/\defn{individual loss}.
\item The aggregate loss \(S\) is a \emph{compound random variable} with
primary (secondary) distribution being claim frequency distribution (severity
distribution).
\end{remark}
We work in the collective risk model henceforth in this section.
\item \label{it:collective-risk-s-pgf}
The pgf of \(S\) is
\[
P_S(t)=\boxed{P_N(P_X(t))}
\]
where \(P_N\) and \(P_X\) are the pgfs of \(N\) and \(X\) respectively.

\item \label{it:collective-risk-s-mgf}
The mgf of \(S\) is
\[
M_S(t)=\expv{\qty(e^t)^S}=P_S(e^t)
=P_N(P_X(e^t))=\boxed{P_N(M_X(t))}
\]
where \(M_X\) is the mgf of \(X\).

\item \label{it:collective-risk-s-mean-var}
By \labelcref{it:cpd-dist-mean-var}, the mean and variance of \(S\) are:
\begin{itemize}
\item \(\expv{S}=\boxed{\expv{N}\expv{X}}\).
\item \(\vari{S}=\boxed{\expv{N}\vari{X}+\vari{N}(\expv{X})^2}\).
\end{itemize}

\item Consider the collective risk model equipped with the setting in
\labelcref{it:nlnp-setting}. Suppose that the given insurance policy has an
ordinary/franchise deducible of \(d\) applied to individual losses. Then, the
probability that a loss results in a payment is
\[
v=\prob{X>d}.
\]
Recall that the total number of payments is
\[
N^P=I_1+I_2+\dotsb+I_{N^L}
\]
where \(N^L\) is the total number of losses.

Now, for any \(i=1,2,\dotsc\), let the per-loss and per-payment variables be
respectively:
\begin{itemize}
\item \(Y^L_i=(X_i-d)_{+}\).
\item \(Y^P_i=(X_i-d|X_i>d)\).
\end{itemize}

\item \label{it:agg-pmt-fmlas}
In this setting, the \defn{aggregate payment} can be calculated in the
following ways:
\begin{itemize}
\item per-loss perspective: \(\text{aggregate payment}=\boxed{Y_1^L+\dotsb+Y_{N^L}^L}\)
\begin{note}
This sums ``payments'' associated to all losses (\(N^L\) of them), including those
with zero value.
\end{note}
\item per-payment perspective: \(\text{aggregate payment}=\boxed{Y_1^P+\dotsb+Y_{N^P}^P}\)
\begin{note}
This sums only positive payments (\(N^P\) of them).
\end{note}
\end{itemize}
Both methods result in the same aggregate payment since the difference between
two expressions is just a sum of ``payments'' of zero value.
\end{enumerate}
\subsection{Convolution}
\begin{enumerate}
\item To obtain more distributional quantities for the aggregate loss \(S\) in
the collective risk model, we need the concept of \emph{convolution}.

\item Let \(X_1,X_2,\dotsc\) be i.i.d.\ continuous random variables with the
same cdf \(F_X\) and pdf \(f_X\). Define
\[
S_n=X_1+\dotsb+X_n
\]
which is a sum of \(n\) i.i.d.\ random variables. Denote its cdf by \(F_X^{*n}\):
\[
F_{X}^{*n}(x)=\prob{X_1+\dotsb+X_n\le x}\quad\text{for any \(x\in\R\)}.
\]
\item \label{it:cts-convolution-steps}
To find the cdf \(F_{X}^{*n}\), we can perform the following recursive
process:
\begin{enumerate}
\item Define \(F_{X}^{*1}=F_{X}\) (the common cdf of \(X_1,X_2,\dotsc\)).
\item Use the following recursive formula to find
\(F_X^{*2},F_X^{*3},\dotsc\), up to \(F_X^{*n}\):
\begin{align*}
&F_{X}^{*k}(x)=\prob{X_1+\dotsb+X_k\le x}\\
&\quad=\int_{-\infty}^{\infty}\prob{X_1+\dotsb+X_k\le x|X_k=y}f_{X}(y)\dd{y}&\text{(law of total probability, continuous case)}\\
&\quad=\int_{-\infty}^{\infty}\prob{X_1+\dotsb+X_{k-1}\le x-y}f_{X}(y)\dd{y}\\
&\quad=\boxed{\int_{-\infty}^{\infty}F_{X}^{*(k-1)}(x-y)f_{X}(y)\dd{y}}
\end{align*}
for any \(k=2,3,\dotsc\).

\begin{note}
The operation here is a special case of \emph{convolution} (mathematical
concept) in the context of probability distributions.
\end{note}
\end{enumerate}

\item \label{it:nonneg-cts-convolution-cdf-pdf-fmlas}
In case \(X_1,X_2,\dotsc\) are \emph{nonnegative} (e.g., when they represent
individual losses in the collective risk model), the recursive formula can be
expressed as:
\[
F_{X}^{*k}(x)=\boxed{\int_{0}^{x}F_{X}^{*(k-1)}(x-y)f_X(y)\dd{y}}
\]
for any \(k=2,3,\dotsc\) (since the integrand is zero when \(y\notin(0,x)\)).

Furthermore, the recursive formula for \emph{pdf} can be obtained by
differentiating both sides of this recursive formula:
\[
f_{X}^{*k}(x)=\boxed{\int_{0}^{x}f_{X}^{*(k-1)}(x-y)f_X(y)\dd{y}}
\]
for any \(k=2,3,\dotsc\), where \(f_X^{*n}\) denotes the pdf of
\(X_1+\dotsb+X_n\).

\item \label{it:nonneg-disc-convolution-cdf-pdf-fmlas}
If \(X_1,X_2,\dotsc\) become \emph{discrete} (and are still nonnegative \&
i.i.d., with the same pmf \(f_X\)), then the recursive formulas in
\labelcref{it:nonneg-cts-convolution-cdf-pdf-fmlas} become:
\[
F_{X}^{*k}(x)=\boxed{\sum_{y=0}^{x}F_{X}^{*(k-1)}(x-y)f_X(y)},\quad x=0,1,\dotsc,
\]
for any \(k=2,3,\dotsc\), and
\[
f_{X}^{*k}(x)=\boxed{\sum_{y=0}^{x}f_{X}^{*(k-1)}(x-y)f_X(y)},\quad x=0,1,\dotsc,
\]
for any \(k=2,3,\dotsc\), where \(f_X^{*n}\) denotes the pmf of
\(X_1+\dotsb+X_n\).
\end{enumerate}
\subsection{Cdf and Probability Function of the Aggregate Loss}
\begin{enumerate}
\item Having the recursive formulas from the concept of convolution, we can
derive the cdf and probability function of the aggregate loss \(S\) in the
collective risk model. Henceforth we let \(p_n=\prob{N=n}\) be the pmf of the
claim frequency \(N\).

\item \label{it:collective-risk-s-cdf}
The cdf of the aggregate loss \(S\) is
\begin{align*}
F_S(x)&=\sum_{n=0}^{\infty}p_n\cdot\prob{S\le x|N=n} &\text{(law of total probability)}\\
&=\sum_{n=0}^{\infty}p_n\cdot\prob{X_1+\dotsb+X_n\le x|N=n} \\
&=\sum_{n=0}^{\infty}p_n\cdot\prob{X_1+\dotsb+X_n\le x} &\text{(by independence assumption)}\\
&=\boxed{\sum_{n=0}^{\infty}p_n\cdot F_{X}^{*n}(x)}.
\end{align*}

\item \label{it:collective-risk-s-probf}
Based on the formula in \labelcref{it:collective-risk-s-cdf}, the probability
function of the aggregate loss \(S\) is thus given by:
\begin{itemize}
\item (\(S\) is continuous): differentiating \faIcon{arrow-right} pdf is
\[f_S(x)=\boxed{\sum_{n=0}^{\infty}p_n\cdot f_X^{*n}(x)},\] where
\(f_X^{*n}\) is the pdf of \(X_1+\dotsb+X_n\).
\item (\(S\) is discrete): pmf is
\[f_S(x)=F_S(x+1)-F_S(x)
=\sum_{n=0}^{\infty}p_n\cdot [F_{X}^{*n}(x+1)-F_{X}^{*n}(x)]
=\boxed{\sum_{n=0}^{\infty}p_n\cdot f_{X}^{*n}(x)},\]
where \(x=0,1,\dotsc\) and \(f_X^{*n}\) is the pmf of \(X_1+\dotsb+X_n\).
\end{itemize}
\end{enumerate}
\subsection{Panjer's Recursion for Discrete Severity}
\begin{enumerate}
\item When the severity is discrete, the aggregate loss in the collective risk
model \(S\) ``matches'' with the setting in
\labelcref{it:cpd-dist-as-total-claims} (just with different notations and
``labels''), so \emph{Panjer's recursion} is applicable for finding the
distribution of \(S\) recursively.

\item \label{it:panjer-recursion-ab0-agg-loss}
By \cref{thm:panjer-recursion-ab0}, when the claim frequency \(N\) is in
the \((a,b,0)\) class, the pmf of \(S\) can be obtained recursively by
\[
{\color{violet}f_S(x)}=\boxed{\frac{1}{1-af_X(0)}\sum_{y=1}^{x}\qty(a+\frac{by}{x})f_X(y){\color{violet}f_S(x-y)}}
\]
for any \(x\ge 1\) (in the support of \(S\)).

\begin{note}
The starting value \(f_S(0)\) can be obtained by
\[
f_S(0)=P_S(0)=P_N(P_X(0))=P_N(f_X(0)).
\]
\end{note}

\item \label{it:panjer-recursion-ab1-agg-loss}
By \cref{thm:panjer-recursion-ab1}, when the claim frequency \(N\) is in
the \((a,b,0)\) class, we have:

\[
{\color{violet}f_S(x)}=\boxed{\frac{1}{1-af_X(0)}\qty{
{\color{orange}[p_1-(a+b)p_0]f_X(x)}+
\sum_{y=1}^{x}\qty(a+\frac{by}{x})f_X(y){\color{violet}f_S(x-y)}}}
\]
for any \(x\ge 1\) (in the support of \(S\)).
\end{enumerate}

\subsection{Method of Rounding}
\begin{enumerate}
\item To apply the recursive formulas in
\labelcref{it:panjer-recursion-ab0-agg-loss,it:panjer-recursion-ab1-agg-loss},
the severity \(X\) needs to be \emph{discrete}. Thus, they are not applicable
for \emph{continuous} severity. Nevertheless, we may still ``apply'' the
formulas by \emph{approximating} the continuous severity distribution by a
discrete distribution (known as \defn{discretization}). Here, we introduce a
way of discretization: \emph{method of rounding}.

\item The \defn{method of rounding} is as follows. Let \(X\) be the original
continuous severity with cdf \(F_X\). Then, we approximate \(X\) by a discrete
random variable \(X'\) defined by:
\[
X'=\begin{cases}
0&\text{w.p. \(f_0=F_X(h/2)\)}; \\
h&\text{w.p. \(f_1=F_X(h+h/2)-F_X(h-h/2)\)}; \\
2h&\text{w.p. \(f_2=F_X(2h+h/2)-F_X(2h-h/2)\)}; \\
\vdots&\vdots \\
jh&\text{w.p. \(f_j=F_X(jh+h/2)-F_X(jh-h/2)\)}; \\
\vdots&\vdots \\
\end{cases}
\]
\begin{center}
\begin{tikzpicture}
\draw[-Latex] (0,0) -- (10,0)
node[right]{\(X\)};
\draw[-Latex] (0,-2) -- (10,-2)
node[right]{\(X'\)};
\draw[<->, brown] (0,0.3) -- (0.5,0.3) node[midway, above]{\(h/2\)};
\foreach \x in {1,...,9} \draw[<->, brown] (\x-0.5,0.3) -- (\x+0.5,0.3) node[midway, above]{\(h\)};
\foreach \x in {2,...,9} \draw[fill] (\x,-2) circle [radius=0.5mm] node[below]{\(\x h\)};
\draw[fill] (0,-2) circle [radius=0.5mm] node[below]{0};
\draw[fill] (1,-2) circle [radius=0.5mm] node[below]{\(h\)};
\draw[thick, decorate,decoration={brace, amplitude=5pt, raise=5pt, mirror}] (0,0) -- (0.5,0);
\foreach \x in {1,...,9}{
\draw[thick, decorate,decoration={brace, amplitude=5pt, raise=5pt, mirror}] (\x-0.5,0) -- (\x+0.5,0);
\draw[thick, ->, magenta] (\x,-1) -- (\x,-1.7);
}
\draw[thick, ->, magenta] (0.25,-1) -- (0,-1.7);
\node[magenta] () at (11,-1) {rounding off};
\node[violet] () at (-1,-0.7) {prob.};
\foreach \x in {1,...,9} \node[violet] () at (\x,-0.7) {\(f_{\x}\)};
\node[violet] () at (0.25,-0.7) {\(f_0\)};
\end{tikzpicture}
\end{center}
\begin{remark}
\item ``w.p.''\ means ``with probability''.
\item We call \(h\) the \defn{span}.
\end{remark}
\end{enumerate}
\subsection{Stop-Loss Premium}
\begin{enumerate}
\item Practically, it is common for an insurance to not apply deductible
\emph{individually} on each loss claimed, but apply it on the losses claimed
for a certain period (say a year) \emph{in aggregate}.

\item Consider an aggregate loss \(S\) (sum of individual losses in a certain
period). An insurance on the aggregate loss \(S\) subject to a deductible \(d\)
is called \defn{stop-loss insurance}. The expected payment per loss of this
insurance is called the \defn{(net) stop-loss premium}:
\[
\expv{(S-d)_{+}}.
\]
\begin{remark}
\item Here we are not required to work in the collective risk model.
\item The insurance ``stops'' the aggregate loss suffered by the policyholder
\faIcon{user} (once \(d\) dollar is reached) \faIcon{arrow-right} hence
``stop-loss insurance''.
\item If one uses the \emph{equivalence principle} (without considering
expenses) to price the insurance policy, the net premium obtained would be
\(\expv{(S-d)_{+}}\) (the expected amount of benefit) \faIcon{arrow-right}
hence ``net stop-loss premium''.
\end{remark}

\item Here we recall some formulas for computing the stop-loss premium
discussed in \cref{sect:dist-quantities}:
\begin{itemize}
\item (\cref{prp:mean-sl-surv})
\[
\expv{(S-d)_{+}}=\boxed{\int_{d}^{\infty}[1-F_S(x)]\dd{x}}
\]
where \(F_S\) is the cdf of \(S\).

\item (\labelcref{it:stop-loss-direct-exp-fmlas}) If \(S\) is continuous with
pdf \(f_S\), then
\[
\expv{(S-d)_{+}}=\boxed{\int_{d}^{\infty}(x-d)f_S(x)\dd{x}}.
\]
If \(S\) is discrete with pmf \(f_S\), then
\[
\expv{(S-d)_{+}}=\boxed{\sum_{s_j>d}^{}(s_j-d)\cdot \prob{S=s_j}}.
\]
\end{itemize}

\item If there is an interval in which the aggregate loss \(S\) has no
probability to lie, the following result may simplify calculations.
\begin{proposition}
\label{prp:agg-loss-linear-interpol}
Suppose that \(\prob{a<S<b}=0\) for some \(a<b\). Then, for any \(d\in[a,b]\),
\[
\expv{(S-d)_{+}}={\color{violet}\frac{b-d}{b-a}\cdot \expv{(S-a)_{+}}}+
{\color{orange}\frac{d-a}{b-a}\cdot \expv{(S-b)_{+}}}.
\]
\begin{note}
This means that the stop-loss premium can be calculated using \emph{linear
interpolation}.
\end{note}
\end{proposition}
\begin{center}
\begin{tikzpicture}
\draw[-Latex] (0,0) -- (10,0);
\draw[fill] (2,0) circle [radius=0.5mm]
node[above, violet]{\(\expv{(S-a)_{+}}\)}
node[below, black]{\(a\)};
\draw[fill] (8,0) circle [radius=0.5mm]
node[above, orange]{\(\expv{(S-b)_{+}}\)}
node[below, black]{\(b\)};
\draw[fill] (6,0) circle [radius=0.5mm]
node[above]{\(\expv{(S-d)_{+}}\)}
node[below]{\(d\)};
\draw[<->, orange] (2,-0.5) -- (6,-0.5)
node[midway, below]{ratio: \(\displaystyle \frac{d-a}{b-a}\)};
\draw[<->, violet] (6,-0.5) -- (8,-0.5)
node[midway, below]{ratio: \(\displaystyle \frac{b-d}{b-a}\)};
\end{tikzpicture}
\end{center}
\begin{pf}
By assumption, the cdf
\[
F_S(x)=F_S(a)
\]
for any \(x\in[a,b)\). Hence,
\begin{align*}
\expv{(S-d)_{+}}&=\int_{d}^{\infty}[1-F_{S}({\color{violet}x})]\dd{x} \\
&=\int_{a}^{\infty}[1-F_{S}(x)]\dd{x}-\int_{a}^{d}[1-F_{S}({\color{violet}a})]\dd{x} \\
&=\expv{(S-a)_{+}}-[1-F_{S}(a)](d-a).
\end{align*}
Setting \(d=b\) here gives
\[
\expv{(S-b)_{+}}=\expv{(S-a)_{+}}-[1-F_{S}(a)](b-a)
\implies 1-F_{S}(a)=\frac{\expv{(S-b)_{+}}-\expv{(S-a)_{+}}}{b-a},
\]
so it follows that
\[
\expv{(S-d)_{+}}=\expv{(S-a)_{+}}-\frac{\expv{(S-b)_{+}}-\expv{(S-a)_{+}}}{b-a}\cdot (d-a)
=\frac{b-d}{b-a}\cdot \expv{(S-a)_{+}}+
\frac{d-a}{b-a}\cdot \expv{(S-b)_{+}}.
\]
\end{pf}

\item When the aggregate loss \(S\) is \emph{discrete}, the stop-loss premiums
are related as follows.
\begin{proposition}
\label{prp:sl-prem-disc-agg-loss}
Suppose that \(S\) is a discrete random variable with support
\(\{0,h,2h,\dotsc\}\) for some fixed \(h>0\). Then, for any \(j\in\N_0\),
\[
\expv{(S-(j+1)h)_{+}}=\expv{(S-jh)_{+}}-h[1-F_{S}(jh)]
\]
where \(F_S\) is the cdf of \(S\).
\end{proposition}
\begin{pf}
\begin{center}
\begin{tikzpicture}
\begin{axis}[domain=0:5, axis lines=middle,
xmin=-0.1, ymax=1.3, ymin=-0.5,
ytick={0.2,0.3,0.5,1}, yticklabels={\(F_S(h)\),\(F_S(2h)\),\(F_S(3h)\),1},
xtick={1,2,3,4}, xticklabels={\(h\),\(2h\),\(3h\),\(4h\)},
xlabel=\(t\), ylabel=\(F_S(t)\),
xlabel style={anchor=west},
ylabel style={anchor=south}]
\addplot[blue, name path=A1, domain=0:1]{0.1};
\addplot[blue, name path=A2, domain=1:2]{0.2};
\addplot[blue, name path=A3, domain=2:3]{0.3};
\addplot[blue, name path=A4, domain=3:4]{0.5};
\addplot[draw=none, name path=B]{1};
\addplot[red, opacity=0.3] fill between[of=A2 and B, soft clip={domain=1:2}];
\addplot[brown, opacity=0.3] fill between[of=A3 and B, soft clip={domain=2:3}];
\addplot[brown, opacity=0.3] fill between[of=A4 and B, soft clip={domain=3:4}];
\draw[blue] (0,0) circle [radius=0.8mm];
\draw[blue, fill] (0,0.1) circle [radius=0.8mm];
\draw[blue, fill] (1,0.2) circle [radius=0.8mm];
\draw[blue, fill] (2,0.3) circle [radius=0.8mm];
\draw[blue, fill] (3,0.5) circle [radius=0.8mm];
\draw[blue] (1,0.1) circle [radius=0.8mm];
\draw[blue] (2,0.2) circle [radius=0.8mm];
\draw[blue] (3,0.3) circle [radius=0.8mm];
\draw[blue] (4,0.5) circle [radius=0.8mm];
\draw[-Latex] (3.5,0.2) -- (2.6,0.5)
node[pos=-0.2]{area = \(\expv{(S-2h)_{+}}\)};
\draw[-Latex] (1.5,-0.2) -- (1.5,0.5)
node[pos=-0.2]{area = \(h[1-F_S(h)]\)};
\draw[very thick, decorate,decoration={calligraphic brace, amplitude=5pt, raise=5pt}] (1,1) -- (4,1)
node[midway, above=0.4cm]{area = \(\expv{(S-h)_{+}}\)};
\end{axis}
\end{tikzpicture}

\begin{tikzpicture}
\begin{axis}[domain=0:5, axis lines=middle,
xmin=-0.1, ymax=1.3, ymin=-0.5,
ytick={0.2,0.3,0.5,1}, yticklabels={\(F_S(h)\),\(F_S(2h)\),\(F_S(3h)\),1},
xtick={1,2,3,4}, xticklabels={\(h\),\(2h\),\(3h\),\(4h\)},
xlabel=\(t\), ylabel=\(F_S(t)\),
xlabel style={anchor=west},
ylabel style={anchor=south}]
\addplot[blue, name path=A1, domain=0:1]{0.1};
\addplot[blue, name path=A2, domain=1:2]{0.2};
\addplot[blue, name path=A3, domain=2:3]{0.3};
\addplot[blue, name path=A4, domain=3:4]{0.5};
\addplot[draw=none, name path=B]{1};
\addplot[red, opacity=0.3] fill between[of=A3 and B, soft clip={domain=2:3}];
\addplot[brown, opacity=0.3] fill between[of=A4 and B, soft clip={domain=3:4}];
\draw[blue] (0,0) circle [radius=0.8mm];
\draw[blue, fill] (0,0.1) circle [radius=0.8mm];
\draw[blue, fill] (1,0.2) circle [radius=0.8mm];
\draw[blue, fill] (2,0.3) circle [radius=0.8mm];
\draw[blue, fill] (3,0.5) circle [radius=0.8mm];
\draw[blue] (1,0.1) circle [radius=0.8mm];
\draw[blue] (2,0.2) circle [radius=0.8mm];
\draw[blue] (3,0.3) circle [radius=0.8mm];
\draw[blue] (4,0.5) circle [radius=0.8mm];
\draw[-Latex] (3.5,0.2) -- (3.6,0.7)
node[pos=-0.2]{area = \(\expv{(S-3h)_{+}}\)};
\draw[-Latex] (1.5,-0.2) -- (2.5,0.5)
node[pos=-0.2]{area = \(h[1-F_S(2h)]\)};
\draw[very thick, decorate,decoration={calligraphic brace, amplitude=5pt, raise=5pt}] (2,1) -- (4,1)
node[midway, above=0.4cm]{area = \(\expv{(S-2h)_{+}}\)};
\end{axis}
\end{tikzpicture}
\end{center}
Based on the geometrical interpretation of expectation (area between the graph
of \(F_S\) and the line \(y=1\), starting at \(t=d\) for \(\expv{(S-d)_{+}}\)),
the result follows readily.
\end{pf}
\end{enumerate}
