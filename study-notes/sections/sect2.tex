\section{Mixing and Conditional Expectation}
\label{sect:mixing-and-ce}
\subsection{Mixing}
\begin{enumerate}
\item Consider \(n\) random variables \(X_1,\dotsc,X_n\) which are all continuous
(all discrete). We can create a new distribution (random variable \(X\)) by
\emph{mixing} them, whose pdf (pmf) is
\[
f_X(x)=p_1f_{X_1}(x)+\dotsb+p_nf_{X_n}(x)
\]
where \(f_{X_i}\) is the pdf (pmf) of \(X_i\) (\(i=1,\dotsc,n\)),
\(p_1,\dotsc,p_n\ge 0\), and \(p_1+\dotsb+p_n=1\). We can interpret the
distribution resulting from mixing by:
\[
X=\begin{cases}
X_1&\text{with probability \(p_1\)};\\
\vdots&\vdots\\
X_n&\text{with probability \(p_n\)}.\\
\end{cases}
\]
\item \label{it:mixing-finite}
By introducing another discrete random variable \(\Lambda\) with support
\(\{\lambda_1,\dotsc,\lambda_n\}\) such that
\[
X_i\eqd (X|\Lambda=\lambda_i)\quad\text{for any \(i=1,\dotsc,n\)},
\]
we can associate \(f_{X_i}(x)\) with \(f_{X|\Lambda}(x|\lambda_i)\) and \(p_i\)
with \(\prob{\Lambda=\lambda_i}\):
\[
f_X(x)=\sum_{i=1}^{n}f_{X|\Lambda}(x|\lambda_i)\prob{\Lambda=\lambda_i}
=\boxed{\sum_{i=1}^{n}f_{X|\Lambda}(x|\lambda_i)p_i}.
\]
\begin{note}
\(f_{X|\Lambda}(x|\lambda_i)\) is the conditional probability function of \(X\)
given \(\Lambda=\lambda_i\).
\end{note}

(The equality holds since each summand is the joint probability function
\(f_{X,\Lambda}(x,\lambda_i)\)i, so summing them up gives the marginal
probability function \(f_{X_i}(x)\).)

This induces mixing of finitely many (all continuous/all discrete) random
variables using (continuous or discrete) \(X\) and discrete \(\Lambda\).

\begin{note}
Here the notion of conditional distribution includes also the case
where one of the two random variables involved is discrete while another is
continuous. (This may not be defined in elementary probability theory, but this
can be allowed through more advanced probability theory. So one may take it as
given.)
\end{note}

\item \label{it:mixing-count-infinite}
To induce mixing of countably infinitely many random variables, we can
introduce a discrete random variable \(\Lambda\) with support
\(\{\lambda_1,\lambda_2,\dotsc\}\). Then, the pdf/pmf of \(X\) can similarly be
written as
\[
f_X(x)=\boxed{\sum_{i=1}^{\infty}f_{X|\Lambda}(x|\lambda_i)p_i}
\]
where \(p_i=\prob{\Lambda=\lambda_i}\) for any \(i\in\N\).

\item \label{it:mixing-uncount-infinite}
To induce mixing of uncountably infinitely many random variables, we can
introduce a continuous random variable \(\Lambda\). Then, the pdf/pmf of \(X\)
can be written as
\[
f_X(x)=\boxed{\int_{-\infty}^{\infty}f_{X|\Lambda}(x|\lambda)f_{\Lambda}(\lambda)\dd{\lambda}}.
\]
(Again here the integrand is the joint probability function
\(f_{X,\Lambda}(x,\lambda)\) so integrating it gives the marginal probability
function \(f_X(x)\).)

\item We can also express the mixing using cdf as suggested by the following
result.
\begin{proposition}
\label{prp:mixing-cdf}
Let \(X\) and \(\Lambda\) be random variables. Then,
\begin{itemize}
\item (\(\Lambda\) is continuous)
\[
F_X(x)=\int_{-\infty}^{\infty}F_{X|\Lambda}(x|\lambda)f_{\Lambda}(\lambda)\dd{\lambda}.
\]
\item (\(\Lambda\) is discrete)
\[
F_X(x)=\sum_{i}^{}F_{X|\Lambda}(x|\lambda_i)p_i
\]
where \(p_i=\prob{\Lambda=\lambda_i}\) and the sum is taken over all \(i\)
where \(p_i>0\).
\end{itemize}
\begin{note}
\(F_X\) is the cdf of \(X\) and \(F_{X|\Lambda}(x|\lambda)\) is the conditional
cdf of \(X\) given \(\Lambda=\lambda\).
\end{note}
\end{proposition}
\begin{pf}
We prove only the continuous case. The discrete case can be proven similarly.
Note that
\begin{align*}
F_X(x)&=\int_{-\infty}^{x}f_X(t)\dd{t}\\
&=\int_{-\infty}^{x}\int_{-\infty}^{\infty}
f_{X|\Lambda}(t|\lambda)f_{\Lambda}(\lambda)\dd{\lambda}\dd{t}\\
&=\int_{-\infty}^{\infty}\int_{-\infty}^{x}
f_{X|\Lambda}(t|\lambda)f_{\Lambda}(\lambda)\dd{t}\dd{\lambda}\\
&=\int_{-\infty}^{\infty}f_{\Lambda}(\lambda){\color{violet}\int_{-\infty}^{x}
f_{X|\Lambda}(t|\lambda)\dd{t}}\dd{\lambda}\\
&=\int_{-\infty}^{\infty}f_{\Lambda}(\lambda){\color{violet}F_{X|\Lambda}(x|\lambda)}\dd{\lambda}.
\end{align*}

\end{pf}
\end{enumerate}
\subsection{Conditional Expectation}
\begin{enumerate}
\item This sections serves as a review on the concept of conditional
expectation, which is covered in STAT2901.
\item Consider again the random variables \(X\) and \(\Lambda\). Recall that
the conditional expectation of \(g(X)\) given \(\Lambda=\lambda\) is
\[
\expv{g(X)|\Lambda=\lambda}=
\begin{cases}
\displaystyle
\int_{-\infty}^{\infty}g(x)f_{X|\Lambda}(x|\lambda)\dd{x}&\text{if \(X\) and \(\Lambda\) are both continuous};\\
\displaystyle
\sum_{i}^{}g(x_i)f_{X|\Lambda}(x_i|\lambda)&\text{if \(X\) and \(\Lambda\) are both discrete}\\
\end{cases}
\]
where the sum is taken over all \(i\) where \(f_{X|\Lambda}(x_i|\lambda)>0\).

Write \(h(\lambda)=\expv{g(X)|\Lambda=\lambda}\).  Then, the conditional
expectation of \(g(X)\) given \(\Lambda\) is denoted by \(\expv{g(X)|\Lambda}\)
and is a \emph{random variable} (as a function of \(\Lambda\)):
\[
h(\Lambda)=\expv{g(X)|\Lambda}
\]
which takes the value \(h(\lambda)\) when \(\Lambda=\lambda\).

\begin{note}
Practically, to obtain an expression of \(\expv{g(X)|\Lambda}\), we can first
find an expression for \(\expv{g(X)|\Lambda=\lambda}\) and replace every
\(\lambda\) by \(\Lambda\).
\end{note}
\item Next, recall that the conditional variance of \(g(X)\) given
\(\Lambda=\lambda\) is
\[
\vari{g(X)|\Lambda=\lambda}
=\expv{(g(X)-\expv{g(X)|\Lambda=\lambda})^{2}|\Lambda=\lambda}
=\expv{g(X)^2|\Lambda=\lambda}-(\expv{g(X)|\Lambda=\lambda})^2.
\]
\begin{note}
Similarly, we can write \(h(\lambda)=\vari{g(X)|\Lambda=\lambda}\), and
\(\vari{g(X)|\Lambda}\) is the random variable \(h(\Lambda)\). Thus,
\[
\vari{g(X)|\Lambda}=\expv{(g(X))^{2}|\Lambda}-(\expv{g(X)|\Lambda})^{2}.
\]
\end{note}
\item Two remarkable results related to conditional expectation and conditional
variance are \emph{law of total expectation} and \emph{law of total variance}.

\item Law of total expectation relates unconditional and conditional
expectations.
\begin{theorem}[Law of total expectation]
\label{thm:lote}
For any function \(g\) and any random variables \(X\) and \(\Lambda\) (where
\(\expv{g(X)}\) is finite),
\[
\expv{\expv{g(X)|\Lambda}}=\expv{g(X)}.
\]
\end{theorem}
\begin{pf}
We only prove for the case where \(X\) and \(\Lambda\) are both continuous.
(The case where both are discrete can be proved similarly.\footnote{A general proof
where both random variables can be arbitrary (as long as \(\expv{g(X)}\) is
finite) is out of scope.}) Let \(h(\lambda)=\expv{g(X)|\Lambda=\lambda}\).
Consider:
\begin{align*}
\expv{g(X)}&=\int_{-\infty}^{\infty}g(x){\color{violet}f_{X}(x)}\dd{x}\\
&=\int_{-\infty}^{\infty}g(x){\color{violet}\int_{-\infty}^{\infty}
f_{X|\Lambda}(x|\lambda)f_{\Lambda}(\lambda)\dd{\lambda}}\dd{x}\\
&=\int_{-\infty}^{\infty}\int_{-\infty}^{\infty}
g(x)f_{X|\Lambda}(x|\lambda)f_{\Lambda}(\lambda)\dd{\lambda}\dd{x}\\
&=\int_{-\infty}^{\infty}f_{\Lambda}(\lambda){\color{orange}\int_{-\infty}^{\infty}
g(x)f_{X|\Lambda}(x|\lambda)\dd{x}}\dd{\lambda}\\
&=\int_{-\infty}^{\infty}{\color{orange}h(\lambda)}f_{\Lambda}(\lambda)\dd{\lambda}\\
&=\expv{h(\Lambda)}\\
&=\expv{\expv{g(X)|\Lambda}}.
\end{align*}
\end{pf}
\item Law of total variance relates unconditional variance and conditional mean
\& variance:
\begin{proposition}[Law of total variance]
\label{thm:lotv}
For any random variables \(X\) (with finite variance) and \(\Lambda\),
\[
\vari{X}=\expv{\vari{X|\Lambda}}+\vari{\expv{X|\Lambda}}.
\]
\end{proposition}
\begin{pf}
Note that
\begin{align*}
\vari{X}&=\expv{X^2}-(\expv{X})^{2}\\
&=\E[{\color{brown}\expv{X^2|\Lambda}}]-\qty(\expv{\expv{X|\Lambda}})^{2}\\
&=\E[{\color{brown}\vari{X|\Lambda}+(\expv{X|\Lambda})^{2}}]-\qty(\expv{\expv{X|\Lambda}})^{2}\\
&=\expv{\vari{X|\Lambda}}+\expv{({\color{violet}\expv{X|\Lambda}})^{2}}
-\qty(\E[{\color{violet}\expv{X|\Lambda}}])^{2}\\
&=\expv{\vari{X|\Lambda}}+\operatorname{Var}({\color{violet}\expv{X|\Lambda}}).
\end{align*}
\end{pf}
\end{enumerate}

